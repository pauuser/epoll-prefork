\section*{ЗАКЛЮЧЕНИЕ}
\addcontentsline{toc}{section}{ЗАКЛЮЧЕНИЕ}

В данной курсовой работе была проанализирована предметная область параллельных корпусов текстов, были даны определения ключевым терминам и описаны способы применения корпуса. 
На основе проведенного анализа были сформулированы требования к наполнению корпуса и его возможностям, разработана ролевая модель. 
В качестве используемой разметки были выбраны морфологическая аннотация и метаразметка. 
В результате проведенного анализа существующих СУБД было выявлено, что наиболее подходящим для решения поставленной задачи видом СУБД является реляционная база данных, развернутая локально.

Были выделены сущности проектируемой базы данных, представлены ER-диаграмма в нотации Чена и диаграмма проектируемой базы данных. 
Для всех таблиц были описаны их атрибуты. 
Разработан алгоритм хранимой функции поиска примеров употребления слова в корпусе, а также алгоритм функции поиска с использованием фильтров. 
Кроме того, был описан формат кеширования данных и хранения логов.

В качестве реляционной СУБД была выбрана PostgreSQL, в качестве кеширующей СУБД Redis. Приложение к базе данных было написано на языке C\# в формате Web API. 
Кроме того, в качестве вспомогательного сервиса к нему было написано приложение на языке Python, занимающееся обработкой лингвистических данных. 
Разработанная система логирования была реализована. 
Была протестирована ролевая модель.

В результате замеров времени и нагрузочного тестирования было установлено, что использование кеширующей СУБД сокращает время повторного получения данных на 25\%. Однако, в связи с необходимостью записи данных в кеширующую СУБД, время получения ответа на новый запрос увеличивается.

Итак, решены следующие задачи:

\begin{itemize}[label=---]
	\item проанализирована предметная область параллельных корпусов текстов, сформированы требования для хранения разметки текстов;
	
	\item проанализированы существующие СУБД;
	
	\item спроектирована БД для хранения и пополнения параллельного корпуса текстов и ролевая модель; 
	
	\item спроектировано кеширование запросов с использованием дополнительной БД, спроектирована систему логирования действий пользователей;
	
	\item выбраны средства реализации системы, реализована спроектированная БД и необходимый интерфейс для взаимодействия с ней;
	
	\item реализована система логирования действий пользователей и проведено тестирование разделения ролей;
	
	\item проведено нагрузочное тестирование, определена зависимость времени исполнения запросов от использования кеширующей БД.
\end{itemize}

Таким образом, цель данной курсовой работы выполнена: разработана база данных для хранения и обработки параллельного корпуса переведённых текстов.

\pagebreak