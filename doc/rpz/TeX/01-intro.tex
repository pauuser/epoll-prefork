\specsection{ВВЕДЕНИЕ}
%\addcontentsline{toc}{section}{ВВЕДЕНИЕ}

Параллельный корпус -- неотъемлемая часть науки о переводе. 
В наши дни параллельный корпус стал неотъемлемой частью уже и самого процесса перевода. Переводчику необходимы ресурсы, предлагающие ему варианты интерпретации исходного текста для подтверждения собственных гипотез. 
На практике больше половины времени, затрачиваемого на перевод, уходит на просмотр вспомогательных материалов~\cite{rura-tranlator-aid-2008}. 
Таким образом, использование параллельного корпуса может существенно ускорить работу переводчика.

Целью данной курсовой работы является разработка базы данных для хранения и обработки параллельного корпуса переведённых текстов.

Для достижения поставленной цели необходимо решить следующие задачи:

\begin{itemize}[label=---]
	\item проанализировать предметную область параллельных корпусов текстов, сформировать требования для хранения разметки текстов;
	
	\item проанализировать существующие СУБД;
	
	\item спроектировать БД для хранения и пополнения параллельного корпуса текстов и ролевую модель;
	
	\item cпроектировать кеширование запросов с использованием дополнительной БД, спроектировать систему логирования действий пользователей;
	
	\item выбрать средства реализации системы, реализовать спроектированную БД и необходимый интерфейс для взаимодействия с ней;
	
	\item реализовать систему логирования действий пользователей и провести тестирование разделения ролей;
	
	\item провести нагрузочное тестирование, определить зависимость времени исполнения запросов от использования кеширующей БД.
\end{itemize}

\pagebreak
