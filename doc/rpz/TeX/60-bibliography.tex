\section*{СПИСОК ИСПОЛЬЗОВАННЫХ ИСТОЧНИКОВ}
\addcontentsline{toc}{section}{СПИСОК ИСПОЛЬЗОВАННЫХ ИСТОЧНИКОВ}

\begingroup
\renewcommand{\section}[2]{}
\begin{thebibliography}{}
	\bibitem{ibatova-lexema-2020}
	Ибатова, А.Ш. Определение и понятие лексемы в лексикологии / А.Ш. Ибатова // Science and world. -- 2020. -- P. 42-43.
	
	\bibitem{rura-tranlator-aid-2008}
	Rura, L., Vandeweghe, W., Montero Perez, M. Designing a parallel corpus as multifunctional translator's aid // Proceedings of the 18th FIT world Congress. -- 2008. -- 11 p.
	
	\bibitem{maltseva-opredelenie-korpusa-2011}
	Мальцева, М.С. К определению термина <<Лингвистический корпус>> и его вариантов / М.С. Мальцева // Социально-экономические явления и процессы. -- 2011. -- №7. -- С. 255-258.
	
	\bibitem{khosla-survey-report-2018}
	Khosla S., Acharya H. A survey report on the existing methods of building a parallel corpus // International Journal of Advanced Research in Computer Science. -- Vol.9, №4. -- 2018. -- P. 13-19.
	
	\bibitem{spyns-theory-and-applications-NLP-2013}
	Essential Speech and Language Technology for Dutch: Results by the STEVIN programme / P. Spyns [et al.] -- The Hague: Springer, 2013. -- 414 p.
	
	\bibitem{szudarski-corpus-linguistics-for-vocab-2017}
	Szudarski P. Corpus Linguistics for Vocabulary: A Guide for Research / P. Szudarski. -- London: Routledge, 2017. -- 238 p.
	
	\bibitem{nelson-building-written-corpus-2010}
	The Routledge Handbook of Corpus Linguistics / A. O'Keeffe [et al.] -- London: Routledge, 2010. -- 712 p.
	
	\bibitem{annotation-russian}
	Виды разметки -- Национальный корпус русского языка. Режим доступа: https://ruscorpora.ru/page/annotation/ (дата обращения: 07.03.2023)
	
	\bibitem{leech-annotation-schemes-1993}
	Leech G. Corpus Annotation Schemes // Literary and Linguistic Computing. -- Vol. 8, №4. -- 1993. -- P. 275-281.
	
	\bibitem{archer-pragmatic-annotation-2008}
	Archer D., Culpeper J., Davies M. Pragmatic Annotation // Corpus Linguistics: An International Handbook. -- Berlin: Mouton de Gruyter, 2008. -- P. 613-641. 
	
	\bibitem{ruscorpora}
	Национальный корпус русского языка. Режим доступа: https://ruscorpora.ru/ (дата обращения: 07.03.2023)
	
	\bibitem{poibeau-MT-2017}
	Poibeau, T. Machine Translation / T. Poibeau. -- Cambridge: the MIT Press, 2017. -- 215 p.
	
	\bibitem{xml}
	Extensible Markup Language (XML). Режим доступа: https://www.w3.org/XML/ (дата обращения: 01.03.2023)
	
	\bibitem{html}
	HTML Standard. Режим доступа: https://html.spec.whatwg.org/multipage/ (дата обращения: 01.03.2023)
	
	\bibitem{pezik-polish-2011}
	Pęzik P., Ogrodniczuk M., Przepiórkowski A. Parallel and spoken corpora in an open repository of Polish language resources. -- 2011. -- 6 p.
	
	\bibitem{tao-ruschi-corpus-2015}
	Тао, Ю. Захаров, В.П. Разработка и использование параллельного корпуса
	русского и китайского языков // Информационные процессы и системы. -- 2015.-- №4. -- С. 18-29.
	
	\bibitem{pavlov-xml-to-sql-2004}
	Павлов, М.Н. Подходы к организации загрузки информации из xml-документов в реляционные базы данных // Известия Орловского Государственного Технического Университета. Серия: Информационные системы и технологии. -- 2004. -- №5(6). -- С. 106-109
	
	\bibitem{krenke-db-design-2005}
	Кренке, Д. Теория и практика построения баз данных / Д. Кренке. -- 9-e изд. -- СПб.: Питер, 2005. -- 859 с.
	
	\bibitem{karpova-db-textbook-2009}
	Карпова, И.П. Базы Данных. Учебное пособие / И. П. Карпова. -- М.: Московский государственный институт электроники и математики (Технический университет), 2009. -- 131 с.
	
	\bibitem{guschin-db-2015}
	Гущин, А.Н. Базы данных: учебно-методическое пособие / А. Н. Гущин. -- 2-е изд., испр. и доп. -- М.-Берлин: Директ-Медиа, 2015. -- 311 с.
	
	\bibitem{avrunev-db-models-2018}
	Аврунев, О.В., Стасышин, В.М. Модели баз данных: учебное пособие / О.Е. Аврунев, В.М. Стасышин. -- Новосибирск: Изд-во НГТУ, 2018. -- 124 c.
	
	\bibitem{shilin-NF-2016}
	Шилин, А.С. Практическая нецелесообразность нормальных форм высокого порядка // Информатика и прикладная математика. -- 2016. -- №. 22. -- С. 116-122.
	
	\bibitem{parfenov-postrelational-db-2016}
	Парфенов, Ю.П. Постреляционные хранилища данных : учеб. пособие / Ю.П. Парфенов. -- Екатеринбург: Изд-во Урал. ун-та, 2016. -- 120 с.
    
    \bibitem{reverso}
    Reverso Context: использование контекстного переводчика. Режим доступа: https://context.reverso.net/перевод/about/ (дата обращения: 10.03.2023)
    
    \bibitem{abbyy-lingvo}
    ABBYY Lingvo Live -- онлайн-словарь. Режим доступа: https://www.lingvolive.com/ru-ru/ (дата обращения: 10.03.2023)
    
    \bibitem{linguee}
    Linguee Dictionary. Режим доступа: https://www.linguee.com/ (дата обращения: 10.03.2023)
    
    \bibitem{postgres}
    PostgreSQL: Documentation. Режим доступа: https://www.postgresql.org/docs/ (дата обращения: 15.04.2023)
    
    \bibitem{taipalus-DBMS-survey}
    Taipalus, T. Database Management System Performance Comparisons: A Systematic Survey / T. Taipalus. -- Jyväskylä: University of Jyväskylä, 2023. -- 32~p.
    
    \bibitem{mysql}
    MySQL Documentation. Режим доступа: https://dev.mysql.com/doc/ (дата обращения: 15.04.2023)
    
    \bibitem{sql-server}
    SQL Server technical documentation. Режим доступа: https://learn.microsoft.com/en-us/sql/sql-server/?view=sql-server-ver16 (дата обращения: 15.04.2023)
    
    \bibitem{mysql-functions}
    CREATE FUNCTION Statement for Loadable Functions. Режим доступа: https://dev.mysql.com/doc/refman/5.7/en/create-function-loadable.html (дата обращения: 15.04.2023)
    
    \bibitem{tiobe}
    TIOBE Index for April 2023. Режим доступа: https://www.tiobe.com/tiobe-index/ (дата обращения: 15.04.2023)
    
    \bibitem{c-sharp}
    Документация по C$\#$. Режим доступа: https://learn.microsoft.com/ru-ru/dotnet/csharp/ (дата обращения: 15.04.2023)
    
    \bibitem{python}
    Python documentation. Режим доступа: https://docs.python.org/3/ (дата обращения: 15.04.2023)
    
    \bibitem{rider}
    Rider: Fast $\&$ powerful cross-platform .NET IDE. Режим доступа: https://www.jetbrains.com/rider/ (дата обращения: 15.04.2023)
    
    \bibitem{pycharm}
    PyCharm: IDE для профессиональной разработки на Python. Режим доступа: https://www.jetbrains.com/ru-ru/pycharm/ (дата обращения: 15.04.2023)
    
    \bibitem{redis}
    Introduction to Redis. Режим доступа: https://redis.io/docs/about/ (дата обращения: 15.04.2023)
    
    \bibitem{aspnet}
    Create web APIs with ASP.NET Core. Режим доступа: https://learn.microsoft.com/en-us/aspnet/core/web-api/?WT.mc\_id=dotnet-35129-website\&view=aspnetcore-7.0 (дата обращения: 15.04.2023)
    
    \bibitem{swagger}
    Swagger Documentation. Режим доступа: https://swagger.io/docs/ (дата обращения: 15.04.2023)
    
    \bibitem{nlog}
    NLog. Flexible \& free open-source logging for .NET. Режим доступа: https://nlog-project.org/ (дата обращения: 15.04.2023)
    
    \bibitem{docker}
    Docker Docs: How to build, share and run applications. Режим доступа: https://docs.docker.com/ (дата обращения: 15.04.2023)
    
    \bibitem{wilkins-logging-2022}
    Wilkins, P. Logging in Action. With Fluentd, Kubernetes and more / Phil Whilkins. -- Shelter Island: Manning, 2022. -- 394 p.
    
    \bibitem{oswind}
    Windows. Режим доступа: https://www.microsoft.com/ru-ru/windows (дата обращения: 15.04.2023)
    
    \bibitem{intel}
    Процессор Intel® Core™ i5-8265U. Режим доступа: https://ark.intel.com/content/www/ru/ru/ark/products/149088/intel-core-i58265u-processor-6m-cache-up-to-3-90-ghz.html (дата обращения: 15.04.2023)
    
    \bibitem{jmeter}
    Apache JMeter - Apache JMeter™. Режим доступа: https://jmeter.apache.org/ (дата обращения: 25.04.2023)
    
\end{thebibliography}
\endgroup

\pagebreak